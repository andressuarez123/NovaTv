---- Laravel commands

* laravel instalation
- composer global require laravel/installer

* Creación de un proyecto
- laravel new novatv


---- Comandos GIT

* Inicializar repositorio
- git init

* Guardar cambios en el stage
- git add .

* Guardar versión
- git commit -m "vesión #1"


---- Comandos artisan

* Artisan: sotware que permite gestionar un programa en laravel desde la línea de comando 

importar el modelo 
use App\Models\User;


* execute migration
- php artisan migrate

* Abrir interfaz de consola
- php artisan tinker

---- Comandos ORM

++++  CREATE
$user->save();


++++ READ
$users = User::all(); //obtiene todos los registros 
$user = User::find(3); // Obtiene un registro específico
$users = User::find([1,2,3]); //Obtiene varios registros
$count = User::count(); //Obtiene la cantidad de usuarios
$user = new User;
        $user->fullname = 'Andres felipe sandoval suarez ';
        $user->email = 'pipe1998sando@gmailcom'; 


crear una migracion:
php artisan make:migration  create_roles_table




  php artisan migrate:fresh           borra toda la informacion
  php artisan migrate:install         Create the migration repository
  php artisan migrate:refresh         Reset and re-run all migrations
  php artisan migrate:reset           Rollback all database migrations
  php artisan migrate:rollback        Rollback the last database migration
  php artisan migrate:status          Show the status of each migration


  
    crear un Seeder 

  php artisan make:seeder RoleSeeder

  Creación de factory : 
php artisan make:factory factories

UPDAte: actualizar informacion 

$user=User::find(id);
$user=imail= 'homero@gmnail.com
$user->save();

delete
$user=::find(id);
$user->delete();


como crear un controlador 

php artisan make::controller  ---->User controller

php artisan make:controler  Test controler 




problemas  y soluciones para abrir en el navegdor
php artisan key:generate


borrar cache 
php artisan optimaize




crear controladores
php artisan make:controller  UserControler --resource
php artisan make::controller -- 



para instalara bootstrap es necesario primero instalar 
instalacion Laravel Mixin :
composer require laravel/ui
php artisan ui bootstrap --auth
npm install && npm run dev 



RUTAS

php artisan route:list 
php artisan r:l
php artisan route:clear
php artiasan route:cache


